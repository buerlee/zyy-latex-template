\documentclass{coursework}

\title{勘探地球物理新进展\\课程作业}%名称(强烈劝退这门课!!!作业太多啦!!!)
\studentName{ZYY}               %姓名
\studentID{175011071}              %学号
\teacherName{陈儒军}               %任课教师(中南地信院的同学注意了,不建议选该老师的这门课)
\major{17级地球物理班}             %专业班级

%%%每个章节分别列出各自的参考文献,采用multibib宏包,具体用法需要百度或goole一下
%\usepackage{multibib}
%\newcites{one}{参考文献(作业一)}


%%%windows系统下如果想用Times New Roman字体,可以uncomment下面三句,Linux下不适用
%\setmainfont{Times New Roman}
%\setmonofont{Courier New}
%\setsansfont{Arial}

\begin{document}
	\maketitle	
	\section{作业一}
	鲁迅说过:“这个世界不只有眼前的苟且,还有明天的苟且,后天的苟且,以及陈老师的4w字作业”\citep{luxun}。
	\subsection{论述航空电磁法仪器进展}
	\citep{Fountain1998,AUKEN201747}。
	
	......
	\subsection{论述地面电磁法仪器进展}
	......
	\subsection{论述大规模高密度电磁探测的意义和作用}
	......
	\subsection{论述频谱激电进展及发展趋势}
	......
	
	\section{作业二}
	\subsection{论述三维MT/AMT在地热勘探中的应用}
	......	
	\subsection{论述三维MT/AMT在活火山研究中的应用}
	......	
	\subsection{论述决定岩矿石频谱激电响应的因素}
	......	
	\subsection{论述频谱激电区分矿与非矿原理}
	......	
	
	\bibliographystyle{model5-names.bst}%参考文献格式,model5-names.bst是爱思唯尔的提供的参考文献格式之一,这里也可以换成其他的
	\bibliography{mybibfile}%把需要用到的参考文献写在mybibfile.bib文件中
\end{document}
